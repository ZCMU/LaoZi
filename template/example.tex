%
% 导言
%
\documentclass[12pt]{book}

%设置页面的宽高
\usepackage{geometry}
% big 32K
\geometry{papersize={204mm, 140mm}, top=16.5mm, bottom=16.5mm, left=19.5mm, right=19.5mm}

%引入xeCJK和垂直字体
\usepackage[BoldFont,SlantFont,CJKchecksingle]{xeCJK}
\newCJKfontfamily\zhsong[RawFeature={vertical:+vert:+vhal}]{SimSun}
\newCJKfontfamily\zhhei[RawFeature={vertical:+vert:+vhal}]{SimHei}
\newCJKfontfamily\zhkai[RawFeature={vertical:+vert:+vhal}]{KaiTi}

%引入割注
\usepackage{../packages/gezhu}
\setgezhulines{2}
\everygezhu{\fontsize{6}{7}\selectfont}
\setgezhuraise{-3pt}

%页面旋转90度
\makeatletter
\usepackage{atbegshi}
\AtBeginShipout{
  \global\setbox\AtBeginShipoutBox\vbox{
    \special{pdf: put @thispage <</Rotate 90>>}
    \box\AtBeginShipoutBox
  }
}

%引入中文数字
\usepackage{zhnumber}

%标题格式
\usepackage{titlesec}
\titleformat{\part}{\Huge\bfseries\selectfont \zhkai}
{}{0pt}{}
\titleformat{\chapter}{\LARGE\bfseries\selectfont \zhhei}
{第\,\zhdigits{\thechapter}\,章}{0pt}{}
\titleformat{\section}{\Large\bfseries\selectfont \zhhei}
{}{0pt}{}

%目录格式
\usepackage{titletoc}
\titlecontents{part}[0pt]{\Huge\bfseries\filright\selectfont \zhkai}
{~}{}
{\titlerule*[1pc]{.}\zhsong\contentspage}
\titlecontents{chapter}[10pt]{\LARGE\bfseries\filright\selectfont \zhhei}
{\zhnumsetup{style={Traditional,Normal}}
\contentspush{第\,\zhdigits{\thecontentslabel}\,章\ }
}{}
{\titlerule*[1pc]{.}\zhsong\contentspage}
\titlecontents{section}[20pt]{\Large\bfseries\filright\selectfont \zhhei}
{~}{}
{\titlerule*[1pc]{.}\zhsong\contentspage}

\renewcommand\contentsname{目录}

%引入页眉页脚
\usepackage{fancyhdr}

%
% 文档
%
\begin{document}
\makeatletter
\gezhu@makespecials

\parskip=15pt

%标题页
\thispagestyle{empty}
\begin{titlepage}
\centering
\vspace*{\stretch{3}}
{\Huge\bfseries 标题}\\
\vspace{\stretch{3}}
{\large 作者}\\
\vspace{\stretch{2}}
\date{}
\end{titlepage}
\newpage

%双面版式的页眉页脚
\fancypagestyle{plain}{ %每章节第一页的格式单独设置
\fancyhf{}
\rfoot{\thepage}
\renewcommand\headrulewidth{0pt}
}
\pagestyle{fancy}
\fancyhf{}
\fancyhead[RE]{\thepage}
%\fancyfoot[RO,RE]{\thepage}
\fancyfoot[RO]{\thepage}
\renewcommand\headrulewidth{0pt}

%目录页
\zhnumsetup{style={Traditional,Financial}}
\pagenumbering{zhdig}
\tableofcontents
\newpage

%正文
\zhnumsetup{style={Traditional,Normal}}
\pagenumbering{zhdig}

%篇
\part*{總論}

%章
\chapter*{导语}
\addcontentsline{toc}{chapter}{导语}

这是导语。

%篇
\part*{上篇}

%章
\chapter{~}

%节
\section{比较}

%进入割注环境
\begin{withgezhu}

%切换字体
\zhsong

%文字
这是割注\gezhu{在正文中插入的『双行』或多行注解}排版测试|{本测试仅供参考}文字。

\end{withgezhu}
\end{document}

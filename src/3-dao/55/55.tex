%
% 第五十五章
%

\chapter{~}

%节
\section{原道德經第十一章}

%进入割注环境
\begin{withgezhu}

%切换字体
\zhsong

%文字

卅(\textcolor{tongjia-color}{三十})\colorbox{adding-color}{楅(\textcolor{tongjia-color}{輻})同一轂,當}其无,
\colorbox{adding-color}{有車}之用\colorbox{adding-color}{也}。
撚(\textcolor{tongjia-color}{埏})\gezhu{乙本此字為左土字旁,右然字}填\gezhu{乙本多一而字}為器,當其无,有填器\colorbox{adding-color}{之用也}。
\colorbox{adding-color}{鑿戶牖},當其无,有\colorbox{adding-color}{室之}用也。
故有之以為利,无之以為用。

\end{withgezhu}

%节
\section{王弼本}

\begin{withgezhu}

\zhsong

三十輻共一轂,當其無,有車之用\gezhu{若斷句為“當其無有,車之用也”,並不符合思想內涵}。
埏填以為器,當其無。有用之器。故有之以為利,無之以為用。

\end{withgezhu}

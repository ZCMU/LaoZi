%
% 第六十一章
%

\chapter{~}

%节
\section{原道德經第十七章}

%进入割注环境
\begin{withgezhu}

%切换字体
\zhsong

%文字

太上,下知有\gezhu{乙本為又字}\colorbox{missing-color}{之。其次},親譽之。
其次,畏之。其下,母(\textcolor{tongjia-color}{侮})之。
信不足,案\gezhu{乙本為安字}有不信。猶呵,其貴言也。
成功遂事,而百省\gezhu{乙本為姓字}胃(\textcolor{tongjia-color}{謂})我自然。

\end{withgezhu}

%节
\section{王弼本}

\begin{withgezhu}

\zhsong

太上,下知有之。其次,親而譽之。其次,畏之。其次,侮之。信不足焉,有不信焉。
悠兮,其貴言。功成事遂,百姓皆謂我自然。

\end{withgezhu}

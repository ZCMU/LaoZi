%
% 第六十四章
%

\chapter{~}

%节
\section{原道德經第二十章}

%进入割注环境
\begin{withgezhu}

%切换字体
\zhsong

%文字

唯與訶\gezhu{乙本為呵字},其相去幾何?美與惡\gezhu{乙本為亞字},其相去何若?
人之\colorbox{adding-color}{所畏},亦不\colorbox{adding-color}{可以不畏人}。
\colorbox{adding-color}{望呵,其未央才(\textcolor{tongjia-color}{哉})}!
眾人巸巸(\textcolor{tongjia-color}{熙熙}),若鄉(\textcolor{tongjia-color}{饗})於大牢,而春登臺。
我泊\gezhu{乙本為博字}焉未佻(\textcolor{tongjia-color}{兆})\gezhu{乙本為垗字},若\colorbox{adding-color}{嬰兒未咳}。
累呵,如\gezhu{乙本為似字}\colorbox{adding-color}{無所歸}。
\colorbox{adding-color}{眾人}皆有\gezhu{乙本為又字}余,我獨遺(\textcolor{tongjia-color}{匱})\gezhu{乙本無此句},
我禺\gezhu{乙本為愚字}人之心也,惷惷(\textcolor{tongjia-color}{沌沌})\gezhu{乙本為湷湷}呵。
鬻(\textcolor{tongjia-color}{俗})\colorbox{adding-color}{人昭昭,我獨如}〇\gezhu{無法拼寫,繁體門裏面一個鵑旁}(\textcolor{tongjia-color}{昏})呵。
鬻(\textcolor{tongjia-color}{俗})人蔡蔡\gezhu{乙本為察察},我獨〇〇\gezhu{兩字為悶字心上加一口}(\textcolor{tongjia-color}{悶悶})呵。
忽\gezhu{乙本為沕字}呵,其若\colorbox{adding-color}{海}。望(\textcolor{tongjia-color}{恍})呵,其若無所止。
\colorbox{adding-color}{眾人皆有以,我獨䦎(\textcolor{tongjia-color}{頑})}以悝(\textcolor{tongjia-color}{俚})\gezhu{乙本為鄙字}。
我\gezhu{乙本為吾字}欲獨異余人,而貴食母。

\end{withgezhu}

%节
\section{王弼本}

\begin{withgezhu}

\zhsong

唯之與阿,相去幾何?善之與惡,相去若何?人之所畏,不可不畏。荒兮,其未央哉!
眾人熙熙,如享大牢,如春登臺。我獨泊兮,其未兆,如嬰兒之未孩。儽儽兮,若無所歸。
眾人皆有余,而我獨如遺。我愚人之心也哉,沌沌兮。俗人昭昭,我獨昏昏。俗人察察,我獨悶悶。
澹兮,其若海。飂兮,若無止。眾人皆有以,而我獨頑似鄙。我獨異於人,而貴食母。

\end{withgezhu}

%
% 第五十四章
%

\chapter{~}

%节
\section{原道德經第十章}

%进入割注环境
\begin{withgezhu}

%切换字体
\zhsong

%文字

\colorbox{adding-color}{載營柏(\textcolor{tongjia-color}{魄})抱一,能毋離(\textcolor{tongjia-color}{離})乎?槫(\textcolor{tongjia-color}{摶})氣至(\textcolor{tongjia-color}{致})柔,}能嬰兒乎?
脩(\textcolor{tongjia-color}{滌})除玄藍(\textcolor{tongjia-color}{鑒})\gezhu{乙本作監字},
能無\gezhu{乙本多一有字}疵乎?
\colorbox{adding-color}{愛民栝(\textcolor{tongjia-color}{治})國,能毋以知(\textcolor{tongjia-color}{智})乎?}
\colorbox{adding-color}{天門啟闔},
\colorbox{adding-color}{能為雌乎?明白四達,能毋以知乎?}\gezhu{此句全部遺失,來自乙本}
生之畜之,生而弗\colorbox{adding-color}{有},
\colorbox{adding-color}{長而弗宰也,是胃(\textcolor{tongjia-color}{謂})玄}德。

\end{withgezhu}

%节
\section{王弼本}

\begin{withgezhu}

\zhsong

載營魄抱一,能無離乎?專氣致柔,能嬰兒乎?滌除玄覽,能無疵乎?
愛民治國,能無知乎?天門開闔\gezhu{避漢景帝劉啟名諱},能無雌乎?明白四達,能無為乎?
生之畜之,生而不有,為而不持\gezhu{後人妄加},長而不宰,是謂玄德。

\end{withgezhu}

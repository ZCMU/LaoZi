%
% 第四十五章
%

\chapter{~}

%节
\section{原道德經第一章}

%进入割注环境
\begin{withgezhu}

%切换字体
\zhsong

%文字

道,可道也,非恒道也。\colorbox{missing-color}{名,可名也,非恒名也}。
無名,萬物之始也;有名,萬物之母也。
\colorbox{adding-color}{故}垣無\gezhu{乙本作恒有二字}欲也,
以其\gezhu{乙本無其字}觀其眇(\textcolor{tongjia-color}{妙});
恒有\gezhu{乙本作又字}欲也,以觀其所曒(\textcolor{tongjia-color}{徼})。
兩者同出,異名同胃(\textcolor{tongjia-color}{謂}),玄之有(\textcolor{tongjia-color}{又})玄,
眾眇(\textcolor{tongjia-color}{妙})之\colorbox{adding-color}{門}。

\end{withgezhu}

%节
\section{王弼本}

\begin{withgezhu}

\zhsong

道,可道,非常道。名,可名,非常名。無名,天地之始;有名,天地之母。
故常無欲,以觀其妙;常有欲,以觀其徼。
此兩者同出而異名,同謂之玄,玄而又玄,眾妙之門。

\end{withgezhu}

%
% 第六十五章
%

\chapter{~}

%节
\section{原道德經第二十一章}

%进入割注环境
\begin{withgezhu}

%切换字体
\zhsong

%文字

孔德之容,唯道是從。道之物,唯望(\textcolor{tongjia-color}{恍})唯忽\gezhu{乙本為淴字}。
\colorbox{adding-color}{忽}\gezhu{乙本為淴字}\colorbox{adding-color}{呵恍}呵,中有\gezhu{乙本為又字}象呵。
望(\textcolor{tongjia-color}{恍})呵忽\gezhu{乙本為淴字}呵,中有物呵。
〇\gezhu{左邊三點水,右邊上幽下子}(\textcolor{tongjia-color}{幽})\gezhu{乙本為幼字,通窈字}呵嗚\gezhu{乙本為冥字}呵,
\gezhu{乙本多一其字}中有請(\textcolor{tongjia-color}{情})吔\gezhu{乙本為呵字}。
其請(\textcolor{tongjia-color}{情})甚真,其中\colorbox{adding-color}{有信}。
自今及古,其名不去,以順眾〇\gezhu{左邊人字旁,右邊一個父,乙本為父字}。
吾何以知眾〇\gezhu{左邊人字旁,右邊一個父,乙本為父字}之然\gezhu{乙本多一也字},以此。

\end{withgezhu}

%节
\section{王弼本}

\begin{withgezhu}

\zhsong

孔德之容,惟道是從。道之為物,惟恍惟惚。惚兮恍兮,其中有象。恍兮惚兮,其中有物。
窈兮冥兮,其中有精。其精甚真,其中有信。自古及今,其名不去,以閱眾甫。
吾何以知眾甫之狀哉,以此。

\end{withgezhu}

%
% 第四十八章
%

\chapter{~}

%节
\section{原道德經第四章}

%进入割注环境
\begin{withgezhu}

%切换字体
\zhsong

%文字

\colorbox{adding-color}{道沖(\textcolor{tongjia-color}{盅}),而用之有(\textcolor{tongjia-color}{又})弗}盈也。
淵呵,始(\textcolor{tongjia-color}{似})萬物之宗。
銼(\textcolor{tongjia-color}{挫})其\colorbox{adding-color}{兌(\textcolor{tongjia-color}{銳})},
解其紛\gezhu{乙本作芬字},和其光,同\colorbox{adding-color}{其塵}。
\colorbox{adding-color}{湛呵似}或存,吾不知\colorbox{adding-color}{其誰之}子也,象帝之先。

\end{withgezhu}

%节
\section{王弼本}

\begin{withgezhu}

\zhsong

道沖,而用之或不盈。淵兮,似萬物之宗。挫其銳,解其紛,和其光,同其塵。
湛兮似或存,吾不知誰之子,象帝之先。

\end{withgezhu}

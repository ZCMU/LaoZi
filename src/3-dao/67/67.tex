%
% 第六十七章
%

\chapter{~}

%节
\section{原道德經第二十二章}

%进入割注环境
\begin{withgezhu}

%切换字体
\zhsong

%文字

曲則金\gezhu{乙本為全字},枉\gezhu{乙本為汪字}則定\gezhu{乙本為正字},
窪則盈,敝\gezhu{乙本為敝下面一個衣}則新,少則得,多則惑。
是以聲(\textcolor{tongjia-color}{聖})\gezhu{乙本為聲的繁體字,下部分缺}人持一,以為天下牧。
不\colorbox{adding-color}{自}視(\textcolor{tongjia-color}{是})故明(\textcolor{tongjia-color}{彰})\gezhu{乙本為章字},
不自見\gezhu{乙本多一也字}故章\gezhu{乙本為明字},不自伐故有功,弗矜故能長。夫唯不爭,故莫能與之爭。
古\colorbox{adding-color}{之所胃(\textcolor{tongjia-color}{謂})曲全者},
\colorbox{adding-color}{豈}\gezhu{乙本為幾字}語才(\textcolor{tongjia-color}{哉})!
誠金\gezhu{乙本為全字}歸之。

\end{withgezhu}

%节
\section{王弼本}

\begin{withgezhu}

\zhsong

曲則全,枉則直,窪則盈,敝則新,少則得,多則惑。是以聖人抱一,為天下式。
不自見故明,不自是故彰,不自伐故有功,不自矜故長。夫唯不爭,故天下莫能與之爭。
古之所謂曲則全者,豈虛言哉!誠全而歸之。

\end{withgezhu}

%
% 第五十六章
%

\chapter{~}

%节
\section{原道德經第十二章}

%进入割注环境
\begin{withgezhu}

%切换字体
\zhsong

%文字

五色使人目明\gezhu{乙本為盲字,有爭議},馳騁田臘(\textcolor{tongjia-color}{獵})使人\colorbox{adding-color}{心發狂},
難得之〇\gezhu{字無法拼寫,見書原文,乙本為貨字}使人之行方(\textcolor{tongjia-color}{妨}),
五味使人之口〇\gezhu{湘,三點水旁換成口字旁,乙本為爽字},五音使人之耳\colorbox{missing-color}{聾}。
是以聲(\textcolor{tongjia-color}{聖})人之治也,為腹不\colorbox{adding-color}{為目},
故去罷\gezhu{乙本為彼字}耳\gezhu{乙本為而取二字}此。

\end{withgezhu}

%节
\section{王弼本}

\begin{withgezhu}

\zhsong

五色令人目盲,五音令人耳聾,五味令人口爽,馳騁畋獵令人心發狂,難得之貨令人行妨。
是以聖人為腹不為目,故去彼取此。

\end{withgezhu}

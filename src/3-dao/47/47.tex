%
% 第四十七章
%

\chapter{~}

%节
\section{原道德經第三章}

%进入割注环境
\begin{withgezhu}

%切换字体
\zhsong

%文字

不上賢,使民不爭。\colorbox{adding-color}{不貴難得之貨,使}民不為\colorbox{adding-color}{盜。不見可欲,使}民不亂。
是以聲(\textcolor{tongjia-color}{聖})\gezhu{乙本为聲字上半部分,且多一人字}之\colorbox{adding-color}{治也,虛其心,實其腹,弱其誌},強其骨。
\colorbox{adding-color}{恒}使民無知無欲也,使\colorbox{adding-color}{夫知(\textcolor{tongjia-color}{智})不敢,弗為而已,則無不治矣}。

\end{withgezhu}

%节
\section{王弼本}

\begin{withgezhu}

\zhsong

不尚賢,使民不爭。不貴難得之貨,使民不為盜。不見可欲,使民心不亂。
是以聖人之治,虛其心,實其腹,弱其誌,強其骨。
常使民無知無欲,使夫智者不敢為也,為無為,則無不治。

\end{withgezhu}

%
% 第六十二章
%

\chapter{~}

%节
\section{原道德經第十八章}

%进入割注环境
\begin{withgezhu}

%切换字体
\zhsong

%文字

故大道廢,案\gezhu{乙本為安字}有仁義。
知(\textcolor{tongjia-color}{智})快\gezhu{乙本為慧字}出,案\gezhu{乙本為安字}有\colorbox{missing-color}{大偽}。
六親不和,案\gezhu{乙本為安字}有\gezhu{乙本為又字}畜\gezhu{乙本為孝字}茲(\textcolor{tongjia-color}{慈})。
邦\gezhu{乙本為國字}家悶(\textcolor{tongjia-color}{昏})亂,案\gezhu{乙本為安字}有貞臣。

\end{withgezhu}

%节
\section{王弼本}

\begin{withgezhu}

\zhsong

大道廢,有仁義。智慧出,有大偽。六親不和,有孝慈。國家昏亂,有忠臣。

\end{withgezhu}

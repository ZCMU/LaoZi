%
% 第五十七章
%

\chapter{~}

%节
\section{原道德經第十三章}

%进入割注环境
\begin{withgezhu}

%切换字体
\zhsong

%文字

龍(\textcolor{tongjia-color}{寵})\gezhu{乙本為弄字}辱若驚,貴大梡\gezhu{乙本為患字}若身。
苛\gezhu{乙本為何字}胃(\textcolor{tongjia-color}{謂})龍(\textcolor{tongjia-color}{寵})\gezhu{乙本為弄字}辱若驚?
龍(\textcolor{tongjia-color}{寵})\gezhu{乙本為弄字}之為下\gezhu{乙本多一也字}。
得之若驚,失\colorbox{adding-color}{之}若驚,
是胃(\textcolor{tongjia-color}{謂})龍(\textcolor{tongjia-color}{寵})\gezhu{乙本為弄字}辱若驚。
何胃(\textcolor{tongjia-color}{謂})貴大梡\gezhu{乙本為患字}若身?
吾所以有大梡\gezhu{乙本為患字}者,為吾有身也;及吾無身,有何梡\gezhu{乙本為患字}?
故貴為身於為天下,若可以〇\gezhu{無法拼寫,走之旁內加一個石字}(\textcolor{tongjia-color}{托})\gezhu{乙本為橐字}天下\colorbox{missing-color}{矣};
愛以身為天下,女(\textcolor{tongjia-color}{如})可以寄天下\gezhu{乙本多一矣字}。

\end{withgezhu}

%节
\section{王弼本}

\begin{withgezhu}

\zhsong

寵辱若驚,貴大患若身。何謂寵辱若驚,寵之下\gezhu{後人篡改}。得之若驚,失之若驚,是謂寵辱若驚。
何謂貴大患若身?吾所以有大患者,為吾有身,及吾無身,吾有何患?
故貴以身為天下,若可寄天下;愛以身為天下,若可托天下。

\end{withgezhu}

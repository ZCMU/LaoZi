%
% 第五十九章
%

\chapter{~}

%节
\section{原道德經第十五章}

%进入割注环境
\begin{withgezhu}

%切换字体
\zhsong

%文字

\colorbox{adding-color}{古之善為道者,微眇(\textcolor{tongjia-color}{妙})玄達},深不可誌(\textcolor{tongjia-color}{識})。
夫唯不可誌(\textcolor{tongjia-color}{識}),故強為之容。
曰\gezhu{乙本無此字}:與(\textcolor{tongjia-color}{豫})呵其若冬\colorbox{adding-color}{涉水。猶呵其若}畏四〇\gezhu{無法拼寫,上面兩個口,下面一個文}(\textcolor{tongjia-color}{鄰})\gezhu{甲本破損此字}。
\colorbox{adding-color}{嚴呵}其若客。渙呵其若淩(\textcolor{tongjia-color}{淩})澤(\textcolor{tongjia-color}{釋})。
〇\gezhu{無法拼寫,王字旁右邊一個口,乙本為沌字}(\textcolor{tongjia-color}{敦})呵其若楃\gezhu{乙本為樸字}。
湷(\textcolor{tongjia-color}{混})\colorbox{adding-color}{呵其若濁}。
〇\gezhu{無法拼寫,三點水旁右邊一個繁體字莊}(\textcolor{tongjia-color}{曠})\gezhu{甲本破損此字}\colorbox{adding-color}{呵其}若浴(\textcolor{tongjia-color}{谷})。
濁而情\gezhu{乙本為靜字}之余\gezhu{乙本為徐字}情\gezhu{乙本為清字},
女(\textcolor{tongjia-color}{安})以重(\textcolor{tongjia-color}{動})之余\gezhu{乙本為徐字}生。
葆(\textcolor{tongjia-color}{保})此道者\colorbox{missing-color}{不}欲盈,夫唯不欲\textcolor{wangbi-color}{盈}\gezhu{乙本無此句},
\colorbox{adding-color}{是以能}〇\gezhu{無法拼寫,上面一個敝,下面一個衣}(\textcolor{tongjia-color}{敝})\gezhu{甲本破損此字}\colorbox{adding-color}{而不}成。

\end{withgezhu}

%节
\section{王弼本}

\begin{withgezhu}

\zhsong

古之善為士者,微妙玄通,深不可識。夫唯不可識,故強為之容。豫焉若冬涉川。猶兮若畏四鄰。
儼兮其若容。渙兮若冰之將釋。敦兮其若樸。曠兮其若谷。混兮其如濁。
孰能濁以靜之徐清,孰能安以久動之徐生。保此道者不欲盈,夫唯不盈,故能蔽不新成。

\end{withgezhu}

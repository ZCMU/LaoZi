%
% 第五十三章
%

\chapter{~}

%节
\section{原道德經第九章}

%进入割注环境
\begin{withgezhu}

%切换字体
\zhsong

%文字

㨁(\textcolor{tongjia-color}{持})而盈之,不\colorbox{adding-color}{若其已。揣而}\gezhu{揣,此字左提手,右短字,通揣}兌(\textcolor{tongjia-color}{銳})之,
\colorbox{adding-color}{不}可長葆(\textcolor{tongjia-color}{保})之\gezhu{乙本缺之字}也。
金玉\colorbox{missing-color}{盈}室,莫之\gezhu{乙本多一能字}守也。
富貴而䮦(\textcolor{tongjia-color}{驕}),自遺咎也。
功述(\textcolor{tongjia-color}{遂})身苪(\textcolor{tongjia-color}{退}),
天\colorbox{adding-color}{之道也}。

\end{withgezhu}

%节
\section{王弼本}

\begin{withgezhu}

\zhsong

持而盈之,不如其已。揣而棁之,不可長保。金玉滿堂\gezhu{疑避諱漢惠帝劉盈的名號而篡改},
莫之能守,富貴而驕,自遺其咎。功遂身退,天之道。

\end{withgezhu}

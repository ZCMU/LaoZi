%
% 第五十八章
%

\chapter{~}

%节
\section{原道德經第十四章}

%进入割注环境
\begin{withgezhu}

%切换字体
\zhsong

%文字

視之而弗見,\colorbox{missing-color}{名}之曰〇\gezhu{無法拼寫,上微下耳,乙本為微字}。
聽之而弗聞,名\gezhu{乙本為命字}之曰希。捪之而弗得,名\gezhu{乙本為命字}之曰夷。
三者不可至(\textcolor{tongjia-color}{致})計(\textcolor{tongjia-color}{詰}),
故〇\gezhu{無法拼寫,口裏面一個束,乙本為左絞絲旁,右君字}(\textcolor{tongjia-color}{混})\colorbox{adding-color}{而為一}。
一者,其上不攸(\textcolor{tongjia-color}{皦})\gezhu{乙本為謬字},
其下不忽(\textcolor{tongjia-color}{昧}),尋尋呵不可名\gezhu{乙本為命字}也,復歸於無物。
是胃(\textcolor{tongjia-color}{謂})無狀之狀,
無物之\colorbox{adding-color}{象,是胃(\textcolor{tongjia-color}{謂})淴(\textcolor{tongjia-color}{忽})望(\textcolor{tongjia-color}{恍})}。
\colorbox{adding-color}{隨而不見其後,迎}而不見其首。
持今之道,以禦今之有,以知古始,是胃(\textcolor{tongjia-color}{謂})道紀。

\end{withgezhu}

%节
\section{王弼本}

\begin{withgezhu}

\zhsong

視之不見名曰夷,聽之不聞名曰希,搏之不得名曰微。此三者不可致詰,故混而為一。
其上不皦,其下不昧,繩繩不可名,復歸於無物。是謂無狀之狀,無物之象,是謂惚恍。
迎之不見其首,隨之不見其後。持古之道\gezhu{後世儒家篡改},以禦今之有,能知古始,是謂道紀。

\end{withgezhu}

%
% 第六十章
%

\chapter{~}

%节
\section{原道德經第十六章}

%进入割注环境
\begin{withgezhu}

%切换字体
\zhsong

%文字

至(\textcolor{tongjia-color}{致})虛極也,守情\gezhu{乙本為靜字}表(\textcolor{tongjia-color}{篤})\gezhu{乙本為督字}也,
萬物旁(\textcolor{tongjia-color}{並})作,吾以觀其復也。
天(\textcolor{tongjia-color}{夫})物云云\gezhu{乙本兩個字相同,無法拼寫,衣字旁右邊一個簡體云},各復歸於其\colorbox{adding-color}{根}。
\textcolor{wangbi-color}{歸根}\gezhu{乙本無此二字}\colorbox{adding-color}{曰靜},靜,是胃(\textcolor{tongjia-color}{謂})復命。
復命常也,知常\gezhu{乙本為其字}明也;不知常,㠵(\textcolor{tongjia-color}{妄})\gezhu{乙本為芒字},
㠵(\textcolor{tongjia-color}{妄})\gezhu{乙本為芒字}作,兇。
知常容,容乃公,公乃王,\colorbox{missing-color}{王}乃天,天乃道,\colorbox{adding-color}{道乃}\textcolor{wangbi-color}{久}。
沕\gezhu{乙本為沒字}身不怠\gezhu{乙本為殆字}。

\end{withgezhu}

%节
\section{王弼本}

\begin{withgezhu}

\zhsong

致虛極,守靜篤。萬物並作,吾以觀復。夫物蕓蕓,各復歸其根。
歸根曰靜,是謂復命。復命曰常,知常曰明;不知常,妄作,兇。
知常容,容乃公,公乃王,王乃天,天乃道,道乃久。沒身不殆。

\end{withgezhu}

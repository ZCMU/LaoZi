%
% 第四十九章
%

\chapter{~}

%节
\section{原道德經第五章}

%进入割注环境
\begin{withgezhu}

%切换字体
\zhsong

%文字

天地不仁,以萬物為芻狗;聲(\textcolor{tongjia-color}{聖})\gezhu{乙本为聲字上半部分}人不仁,
\colorbox{missing-color}{以}百省(\textcolor{tongjia-color}{姓})\colorbox{adding-color}{為芻}狗;
天地\colorbox{adding-color}{之間,其}猶橐\gezhu{義為口袋}籥\gezhu{義為笛狀樂器}與?
虛而不淈(\textcolor{tongjia-color}{屈}),
蹱(\textcolor{tongjia-color}{動})\gezhu{乙本作勭字}而俞(\textcolor{tongjia-color}{愈})出。
多聞數窮,不若守於中。

\end{withgezhu}

%节
\section{王弼本}

\begin{withgezhu}

\zhsong

天地不仁,以萬物為芻狗;聖人不仁,以百姓為芻狗;天地之間,其猶橐籥乎?虛而不屈,動而愈出。
多言數窮,不如守中。

\end{withgezhu}

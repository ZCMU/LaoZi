%
% 第六十三章
%

\chapter{~}

%节
\section{原道德經第十九章}

%进入割注环境
\begin{withgezhu}

%切换字体
\zhsong

%文字

絕聲(\textcolor{tongjia-color}{聖})\gezhu{乙本為聲的繁體字上半部分}棄知(\textcolor{tongjia-color}{智}),
\gezhu{乙本多一而字}民利百負\gezhu{乙本為倍字}。
絕仁棄義,\gezhu{乙本多一而字}民復畜\gezhu{乙本為孝字}茲(\textcolor{tongjia-color}{慈})。
絕巧棄利,盜賊無有。此三言也,以為文未足,故令之有所屬。
見素報\colorbox{adding-color}{樸,少私而寡欲}。\colorbox{adding-color}{絕學無憂}。

\end{withgezhu}

%节
\section{王弼本}

\begin{withgezhu}

\zhsong

絕聖棄智,民利百倍。絕仁棄義,民復慈孝。絕巧棄利,盜賊無有。此三者,以為文不足,故令有所屬。
見素報樸,少私寡欲。絕學無憂。

\end{withgezhu}

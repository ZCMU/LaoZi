%
% 第五十章
%

\chapter{~}

%节
\section{原道德經第六章}

%进入割注环境
\begin{withgezhu}

%切换字体
\zhsong

%文字

浴(\textcolor{tongjia-color}{谷})\gezhu{義為中虛}神\colorbox{adding-color}{不}死,
是胃(\textcolor{tongjia-color}{謂})玄牝,
玄牝之門,是胃(\textcolor{tongjia-color}{謂})\colorbox{adding-color}{天}地之根。
綿綿呵\gezhu{乙本多一其字}若存,用之不堇(\textcolor{tongjia-color}{勤})\gezhu{義為無盡}。

\end{withgezhu}

%节
\section{王弼本}

\begin{withgezhu}

\zhsong

谷神不死,是謂玄牝,玄牝之門,是謂天地根。綿綿若存,用之不勤。

\end{withgezhu}

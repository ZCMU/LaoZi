%
% 第六十六章
%

\chapter{~}

%节
\section{原道德經第二十四章}

%进入割注环境
\begin{withgezhu}

%切换字体
\zhsong

%文字

炊(\textcolor{tongjia-color}{企})者不立,自視(\textcolor{tongjia-color}{是})不章(\textcolor{tongjia-color}{彰}),
\colorbox{adding-color}{自}見者不明,自伐者無功,自矜者不長。
其在道\gezhu{乙本多一也字},曰〇\gezhu{左米,右余}(\textcolor{tongjia-color}{余})食贅行,物或惡\gezhu{乙本為亞字}之,
故有欲(\textcolor{tongjia-color}{裕})者\colorbox{adding-color}{弗}居。

\end{withgezhu}

%节
\section{王弼本}

\begin{withgezhu}

\zhsong

企者不立,跨者不行\gezhu{後人加}。自見者不明,自是者不彰,自伐者無功,自矜者不長。
其在道也,曰余食贅行,物或惡之,故有道者不處。

\end{withgezhu}

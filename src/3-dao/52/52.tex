%
% 第五十二章
%

\chapter{~}

%节
\section{原道德經第八章}

%进入割注环境
\begin{withgezhu}

%切换字体
\zhsong

%文字

上善治(\textcolor{tongjia-color}{似})\gezhu{乙本作如字,且有爭議}水,
水善利萬物而有靜\gezhu{乙本作爭字}。
居眾\gezhu{乙本多一人字}之所惡,故幾於道矣。
居善地,心善潚(\textcolor{tongjia-color}{淵}),予善,信\gezhu{乙本這兩句為予善天,言善信},
正(\textcolor{tongjia-color}{政})善治,事善能,蹱(\textcolor{tongjia-color}{動})善時。
夫唯不靜(\textcolor{tongjia-color}{爭}),故無尤。

\end{withgezhu}

%节
\section{王弼本}

\begin{withgezhu}

\zhsong

上善若水,水善利萬物而不爭。居眾人之所惡,故幾於道。
居善地,心善淵,與善仁\gezhu{仁非道家之用詞,乃後人篡改。},言善信,正善治,事善能,動善時。
夫唯不爭,故無尤。

\end{withgezhu}

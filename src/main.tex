%
% 导言
%
\documentclass[12pt]{book}

%设置页面的宽高
\usepackage{geometry}
\geometry{papersize={210mm,148mm}}

%引入xeCJK和垂直字体
\usepackage[BoldFont,SlantFont,CJKchecksingle]{xeCJK}
\newCJKfontfamily\zhsong[RawFeature={vertical:+vert:+vhal}]{SimSun}
\newCJKfontfamily\zhhei[RawFeature={vertical:+vert:+vhal}]{SimHei}
\newCJKfontfamily\zhkai[RawFeature={vertical:+vert:+vhal}]{KaiTi}

%引入割注
\usepackage{../packages/gezhu}
\setgezhulines{2}
\everygezhu{\fontsize{6}{7}\selectfont}
\setgezhuraise{-3pt}

%页面旋转90度
\makeatletter
\usepackage{atbegshi}
\AtBeginShipout{
  \global\setbox\AtBeginShipoutBox\vbox{
    \special{pdf: put @thispage <</Rotate 90>>}
    \box\AtBeginShipoutBox
  }
}

%引入中文数字
\usepackage{zhnumber}

%标题格式
\usepackage{titlesec}
\titleformat{\part}{\Huge\bfseries\selectfont \zhkai}
{}{0pt}{}
\titleformat{\chapter}{\LARGE\bfseries\selectfont \zhhei}
{第\,\zhdigits{\thechapter}\,章}{0pt}{}
\titleformat{\section}{\Large\bfseries\selectfont \zhhei}
{}{0pt}{}

%目录格式
\usepackage{titletoc}
\titlecontents{part}[0pt]{\Huge\bfseries\filright\selectfont \zhkai}
{~}{}
{\titlerule*[1pc]{.}\zhsong\contentspage}
\titlecontents{chapter}[10pt]{\LARGE\bfseries\filright\selectfont \zhhei}
{\zhnumsetup{style={Traditional,Normal}}
\contentspush{第\,\zhdigits{\thecontentslabel}\,章\ }
}{}
{\titlerule*[1pc]{.}\zhsong\contentspage}
\titlecontents{section}[20pt]{\Large\bfseries\filright\selectfont \zhhei}
{~}{}
{\titlerule*[1pc]{.}\zhsong\contentspage}

\renewcommand\contentsname{目录}

%引入页眉页脚
\usepackage{fancyhdr}

%
% 文档
%
\begin{document}
\makeatletter
\gezhu@makespecials

\parskip=15pt

%标题页
\thispagestyle{empty}
\begin{titlepage}
\centering
\vspace*{\stretch{3}}
{\Huge\bfseries 老子比對}\\
\vspace{\stretch{3}}
{\large 袁昕~編}\\
\vspace{\stretch{2}}
\date{}
\end{titlepage}
\newpage

%双面版式的页眉页脚
\fancypagestyle{plain}{ %每章节第一页的格式单独设置
\fancyhf{}
\fancyfoot[RO,RE]{\thepage}
}
\pagestyle{fancy}
\fancyhf{}
\fancyfoot[RO,RE]{\thepage}

%目录页
\zhnumsetup{style={Traditional,Financial}}
\pagenumbering{zhdig}
\tableofcontents
\newpage

%正文
\zhnumsetup{style={Traditional,Normal}}
\pagenumbering{zhdig}

%引言
%
% 引言
%

%篇
\part*{引言}

%章
\chapter*{導語}
\addcontentsline{toc}{chapter}{導語}

《道德經》一書,世傳版本已證實經後人篡改而嚴重失真。
一九七三年,湖南長沙馬王堆漢墓出土帛書《老子》甲、乙本(合稱帛本《老子》),
距今兩千多年的歷史;
一九九三年湖北荊門郭店楚墓中又出土了一個版本更早的竹簡本《老子》,
通過比對,馬王堆漢墓出土的帛本《老子》,也就是後人常說的《道德經》或者《老子五千言》,
是至今世人篡改最少的版本。

\noindent
帛書《老子》順序和傳世本《道德經》不同,《德經》在前,《道經》在後。
本書正文按此順序編排,並根據三個版本相互勘校和比較。
如帛書老子因埋藏而被損壞之字,甲本殘壞則據乙本補,用淺綠色的背景文字表示;
乙本殘壞則據甲本補,用淺紅色背景文字表示;
乙本和甲本不同之處用割註表示;
甲乙版本共同殘壞則據王本或其它今本補,用紅色的文字表示。
帛書老子原有之衍文脫句等錯誤,錄文不刪不補,仍照原文寫成今字,只在割註予以辯證說明。
帛書甲、乙本經文中均有假借字和古字,勘校時除將按原形寫出後,
在其字下註明當用之本字和今字,皆用括弧括起,用藍色的文字以示區分。


%德篇
%
% 德篇
%

%篇
\part*{德篇}

%章
%
% 第一章
%

\chapter{~}

%节
\section{原道德經第三十八章}

%进入割注环境
\begin{withgezhu}

%切换字体
\zhsong

%文字

\colorbox{adding-color}{上德不德,是以有德;下德不失德,是以無}德。
上德無\colorbox{adding-color}{為而}無以為也。
上仁為之\colorbox{adding-color}{而無}以為也。
上義\gezhu{乙本義字作德字}為之而有以為也。
上禮\colorbox{adding-color}{為之而莫之應也,則}
攘臂而乃(\textcolor{tongjia-color}{扔})之
\gezhu{德仁義禮四层递减}。
故失道而後德,失德而後仁,失仁而後義,
\colorbox{adding-color}{失義而後禮。}
\colorbox{adding-color}{夫禮者,}
\colorbox{adding-color}{忠信之泊(\textcolor{tongjia-color}{薄})也},
而亂之首也。
\colorbox{adding-color}{前識者},道之華也,而愚之首也。
是以大丈夫居\colorbox{missing-color}{其厚而不}居其泊(\textcolor{tongjia-color}{薄});
居其實而不居其華。故去皮(\textcolor{tongjia-color}{彼})\gezhu{乙本多一而字}取此。

\end{withgezhu}

%节
\section{王弼本}

\begin{withgezhu}

\zhsong

上德不德、是以有德;下德不失德,是以無德。
上德無為而無以為。下德為之而有以為\gezhu{此句為後人妄加}。
上仁為之而無以為。上義為之而有以為。
上禮為之而莫之應,則攘臂而扔之。
故失道而後德,失德而後仁,失仁而後義,失義而後禮。
夫禮者,忠信之薄而亂之首。前識者,道之華也,而愚之始。
是以大丈夫處其厚不居其薄;處其實不居其華。故去彼取此。

\end{withgezhu}

%
% 第二章
%

\chapter{~}

%节
\section{原道德經第三十九章}

%进入割注环境
\begin{withgezhu}

%切换字体
\zhsong

%文字

昔之\gezhu{乙本缺之字}得一者,
天得一以清,地得\colorbox{adding-color}{一}以寧,
神得一以霝(\textcolor{tongjia-color}{靈}),
浴(\textcolor{tongjia-color}{谷})得一以\gezhu{乙本缺以字}盈,
侯\colorbox{adding-color}{王得一}而\gezhu{乙本缺而字}以為\colorbox{adding-color}{天下}正。
其致(\textcolor{tongjia-color}{誡})之\gezhu{乙本作至字,缺之字}也,
胃(\textcolor{tongjia-color}{謂})天毋已清將恐\colorbox{adding-color}{裂}\gezhu{乙本作蓮字,通裂},
胃(\textcolor{tongjia-color}{謂})\gezhu{乙本缺胃字}地毋\colorbox{adding-color}{已寧}將恐\colorbox{adding-color}{發},
胃(\textcolor{tongjia-color}{謂})\gezhu{乙本缺胃字}神毋\colorbox{missing-color}{已霝(\textcolor{tongjia-color}{靈})}\textcolor{wangbi-color}{將}\colorbox{adding-color}{恐歇},
胃(\textcolor{tongjia-color}{謂})\gezhu{乙本缺胃字}浴(\textcolor{tongjia-color}{谷})毋已\colorbox{missing-color}{盈}將恐渴(\textcolor{tongjia-color}{竭}),
胃(\textcolor{tongjia-color}{謂})\gezhu{乙本缺胃字}侯王毋已貴\colorbox{adding-color}{以高將恐厥(\textcolor{tongjia-color}{蹶})}。
故必貴而\gezhu{乙本缺而字}以賤為本,必高矣而以下為基。
夫是以侯王自胃(\textcolor{tongjia-color}{謂})孤寡不榖\gezhu{榖原文為左半個字,無法拼寫}(\textcolor{tongjia-color}{榖})。
此其\colorbox{adding-color}{賤之本與,非也}?
故致\gezhu{乙本作至字}數與(\textcolor{tongjia-color}{譽})\gezhu{乙本作輿字}無與(\textcolor{tongjia-color}{譽})\gezhu{乙本作輿字}。
是故不欲\colorbox{adding-color}{祿祿}若玉,硌\colorbox{adding-color}{硌若石}。

\end{withgezhu}

%节
\section{王弼本}

\begin{withgezhu}

\zhsong

昔之得一者,天得一以清,地得一以寧,神得一以靈,谷得一以盈,
萬物得一以生\gezhu{此句為後人妄加},侯王得一以為天下貞。
其致之,天無以清將恐裂,地無以寧將恐發,神無以靈將恐歇,谷無以盈將恐竭,
萬物無以生將恐滅\gezhu{此句為後人妄加},侯王無以貴高將恐蹶。
故貴以賤為本,高以下為基。是以侯王自謂孤寡不榖。此非以賤為本邪,非乎?
故致數輿無輿。不欲琭琭如玉,珞珞如石。

\end{withgezhu}

%
% 第三章
%

\chapter{~}

%节
\section{原道德經第四十一章}

%进入割注环境
\begin{withgezhu}

%切换字体
\zhsong

%文字

上\textcolor{wangbi-color}{士聞}道,堇(\textcolor{tongjia-color}{勤})能用之。
中士聞道,若存若亡。下士聞道,大笑之。弗笑,\textcolor{wangbi-color}{不足}以為道。
是以建言有之曰:明道如費(\textcolor{tongjia-color}{昧}),進道如退,夷道如類\gezhu{平坦的道好像不平坦}。
上德如浴(\textcolor{tongjia-color}{谷}),大白如辱\gezhu{汙垢之意}。
廣德如不足,建德如(\textcolor{tongjia-color}{偷})。質\textcolor{wangbi-color}{真如渝}。
大方無禺(\textcolor{tongjia-color}{隅}),大器免成。
大音希聲,天象無刑(\textcolor{tongjia-color}{形}),道褒無名。
\textcolor{wangbi-color}{夫唯}道,善\textcolor{wangbi-color}{始且善成}。

\end{withgezhu}

%节
\section{註記}

\begin{withgezhu}

\zhsong

甲本此章經文大量殘毀,僅留乙本最後一句“道”和“善”二字。
今本第四十章與第四十一章順序顛倒,經研究,帛書《老子》順序正確。

\end{withgezhu}

%节
\section{王弼本}

\begin{withgezhu}

\zhsong

上士聞道,勤而行之。中士聞道,若存若亡。下士聞道,大笑。不笑,不足以為道。
故建言有之:明道如昧,進道若退,夷道若類。
上德若谷,大白若辱。廣德若不足,建德若偷。質真若渝。
大方無隅,大器免成\gezhu{後人傳訛為大器晚成},大音希聲,大象無形,道隱無名。
夫唯道,善貸且成。

\end{withgezhu}

%
% 第四章
%

\chapter{~}

%节
\section{原道德經第四十章}

%进入割注环境
\begin{withgezhu}

%切换字体
\zhsong

%文字

\colorbox{adding-color}{反也者},道之動也;
\colorbox{missing-color}{弱也}者,道之用也。
天\colorbox{adding-color}{下之物生於有,有生\textcolor{wangbi-color}{於}无}。

\end{withgezhu}

%节
\section{王弼本}

\begin{withgezhu}

\zhsong

反者,道之動;弱者,道之用。天下萬物生於有,有生於无。

\end{withgezhu}

%
% 第五章
%

\chapter{~}

%节
\section{原道德經第四十二章(未校註)}

%进入割注环境
\begin{withgezhu}

%切换字体
\zhsong

%文字

道生一,一生二,二生三,三生萬物。萬物負陰而抱陽,沖氣以為和。
人之所惡,唯孤寡不榖,而王公以自名也。故物或益之而損,損之而益。
古人之所教,亦我而教人。故強梁者不得死,我將以為\textcolor{tongjia-color}{學父}。

\end{withgezhu}

%
% 第六章
%

\chapter{~}

%节
\section{原道德經第四十三章(未校註)}

%进入割注环境
\begin{withgezhu}

%切换字体
\zhsong

%文字

天下之至柔,馳騁於天下之至堅。無有入於無間。吾是以知無為之有益也。
不言之教,無為之益,天下希能及之矣。

\end{withgezhu}

%
% 第七章
%

\chapter{~}

%节
\section{原道德經第四十四章(未校註)}

%进入割注环境
\begin{withgezhu}

%切换字体
\zhsong

%文字

名與身孰親?身與貨孰多?得與亡孰病?甚愛必大費,多藏必厚亡。
故知足不辱,知止不殆,可以長久。

\end{withgezhu}

%
% 第八章
%

\chapter{~}

%节
\section{原道德經第四十五章(未校註)}

%进入割注环境
\begin{withgezhu}

%切换字体
\zhsong

%文字

大成若缺,其用不弊,大盈若\textcolor{tongjia-color}{盅},其用不窮。
大直如詘,大巧如拙,\textcolor{tongjia-color}{大贏如絀}。
躁勝寒,靜勝熱,清靜可以為天下正。

\end{withgezhu}

%
% 第九章
%

\chapter{~}

%节
\section{原道德經第四十六章(未校註)}

%进入割注环境
\begin{withgezhu}

%切换字体
\zhsong

%文字

天下有道,卻走馬以糞。天下無道,戎馬生於郊。
罪莫大於可欲,禍莫大於不知足,\textcolor{tongjia-color}{咎莫憯於欲得}。
故知足之足,恒足矣。

\end{withgezhu}

%
% 第十章
%

\chapter{~}

%节
\section{原道德經第四十七章(未校註)}

%进入割注环境
\begin{withgezhu}

%切换字体
\zhsong

%文字

不出於戶,以知天下。不窺於牖,以知天道。其出彌遠者,其知彌少。
是以聖人不行而知,不見而明,弗為而成。

\end{withgezhu}

%
% 第十一章
%

\chapter{~}

%节
\section{原道德經第四十八章(未校註)}

%进入割注环境
\begin{withgezhu}

%切换字体
\zhsong

%文字

為學者日益,\textcolor{tongjia-color}{聞道}者日損。
損之又損,以至於無為,無為而無以為。
取天下也,恒無事,及其有事也,不足以取天下。

\end{withgezhu}

%
% 第十二章
%

\chapter{~}

%节
\section{原道德經第四十九章(未校註)}

%进入割注环境
\begin{withgezhu}

%切换字体
\zhsong

%文字

聖人恒無心,以百姓之心為心。善者善之,不善者亦善之,德善也。
信者信之,不信者亦信之,德信也。聖人之在天下也,歙歙焉,為天下渾心。
百姓皆註其耳目焉,\textcolor{tongjia-color}{聖人皆孩子}。

\end{withgezhu}

%
% 第十三章
%

\chapter{~}

%节
\section{原道德經第五十章(未校註)}

%进入割注环境
\begin{withgezhu}

%切换字体
\zhsong

%文字

出生入死。生之徒十有三,死之徒十有三,而民生生,動皆之死地之十有三。
夫何故?以其生生也。蓋聞攝其生者,陵行不避兕虎,入軍不被甲兵。
兕無所投其角,虎無所措其爪,兵無所容其刃,夫何故也?以其無死地焉。

\end{withgezhu}

%
% 第十四章
%

\chapter{~}

%节
\section{原道德經第五十一章(未校註)}

%进入割注环境
\begin{withgezhu}

%切换字体
\zhsong

%文字

道生之而德畜之,物形之而器成之,是以萬物尊道而貴德。
道之尊也,德之貴也,夫莫之爵也,而恒自然也。
道生之、畜之、長之、育之、亭之、毒之、養之、覆之。
生而弗有,為而弗持,長而弗宰,是謂玄德。

\end{withgezhu}

%
% 第十五章
%

\chapter{~}

%节
\section{原道德經第五十二章(未校註)}

%进入割注环境
\begin{withgezhu}

%切换字体
\zhsong

%文字

天下有始,以為天下母。既得其母,以知其子,復守其母,沒身不殆。
\textcolor{tongjia-color}{塞其㙂},閉其門,終身不勤。
\textcolor{tongjia-color}{啟其㙂},濟其事,終身不救。

\end{withgezhu}

%
% 第十六章
%

\chapter{~}

%节
\section{原道德經第五十三章(未校註)}

%进入割注环境
\begin{withgezhu}

%切换字体
\zhsong

%文字

\textcolor{tongjia-color}{使我介有知},行於大道,唯施是畏。
大道甚夷,民甚好徑,朝甚除,田甚蕪,倉甚虛。
服文采,帶利劍,厭飲食,而資財有余。
\textcolor{tongjia-color}{是謂盜竽},非道也哉!

\end{withgezhu}

%
% 第十七章
%

\chapter{~}

%节
\section{原道德經第五十四章(未校註)}

%进入割注环境
\begin{withgezhu}

%切换字体
\zhsong

%文字

善建者不拔,善抱著不脫,子孫以祭祀不絕。
修之身,其德乃真。修之家,其德乃余。修之鄉,其德乃長。修之國,其德乃豐。修之天下,其德乃博。
以身觀身,以家觀家,以鄉觀鄉,\textcolor{tongjia-color}{以邦觀邦},以天下觀天下。
吾何以知天下之然哉?以此。

\end{withgezhu}

%
% 第十八章
%

\chapter{~}

%节
\section{原道德經第五十五章(未校註)}

%进入割注环境
\begin{withgezhu}

%切换字体
\zhsong

%文字

含德之厚者,比於赤子。\textcolor{tongjia-color}{蜂蠆虺蛇不蜇,攫鳥猛獸不搏}。
骨弱筋柔而握固,未知牝牡之會而朘怒,精之至也。終日號而不嗄。和之至也。
和曰常,知常曰明,益生曰祥,心使氣曰強。物壯則老,謂之不道,不道早已。

\end{withgezhu}

%
% 第十九章
%

\chapter{~}

%节
\section{原道德經第五十六章(未校註)}

%进入割注环境
\begin{withgezhu}

%切换字体
\zhsong

%文字

知者弗言,言者弗知。塞其㙂,閉其門,和其光,同其塵,
\textcolor{tongjia-color}{挫其銳而解其紛},是謂玄同。
故不可得而親,亦不可得而疏;不可得而利,亦不可得而害;
不可得而貴,亦不可得而賤;故為天下貴。

\end{withgezhu}

%
% 第二十章
%

\chapter{~}

%节
\section{原道德經第五十七章(未校註)}

%进入割注环境
\begin{withgezhu}

%切换字体
\zhsong

%文字

以正之治國,以奇治兵,以無事取天下。吾何以知其然也哉?
夫天下多忌諱,而民彌貧。民多利器,而邦家滋昏。人多知巧,而奇物滋起。
\textcolor{tongjia-color}{法物滋彰},而盜賊多有。
是以聖人之言曰:我無為而民自化,我好靜而民自正,我無事而民自富,
\textcolor{tongjia-color}{我欲不欲而民自樸}。

\end{withgezhu}

%
% 第二十一章
%

\chapter{~}

%节
\section{原道德經第五十八章(未校註)}

%进入割注环境
\begin{withgezhu}

%切换字体
\zhsong

%文字

其政悶悶,其民惇惇。其政察察,\textcolor{tongjia-color}{其民㹟㹟}。
禍,福之所倚;福,禍之所伏,孰知其極?
其無正也,正復為奇,善復為妖,人之迷也,其日固久矣。
是以方而不割,兼而不剌,直而不肆,光而不燿。

\end{withgezhu}

%
% 第二十二章
%

\chapter{~}

%节
\section{原道德經第五十九章(未校註)}

%进入割注环境
\begin{withgezhu}

%切换字体
\zhsong

%文字

治人事天莫若嗇。夫惟嗇,是以早服。早服是謂重積德。重積德則無不克,無不克則莫知其極。
莫知其極,可以有國。有國之母,可以長久。是謂深根固柢,長生久視之道也。

\end{withgezhu}

%
% 第二十三章
%

\chapter{~}

%节
\section{原道德經第六十章(未校註)}

%进入割注环境
\begin{withgezhu}

%切换字体
\zhsong

%文字

治大國若烹小鮮,以道蒞天下,其鬼不神。非其鬼不神也,其神不傷人也。
非其神不傷人也,聖人亦弗傷也。夫兩不相傷,故德交歸焉。

\end{withgezhu}

%
% 第二十四章
%

\chapter{~}

%节
\section{原道德經第六十一章(未校註)}

%进入割注环境
\begin{withgezhu}

%切换字体
\zhsong

%文字

\textcolor{tongjia-color}{大邦者下流也},天下之牝也。天下之交也,牝恒以靜勝牡。
為其靜也,故宜為下。\textcolor{tongjia-color}{大邦以下小邦},則取小邦。
小邦以下大邦,則取於大邦。故或下以取,或下而取。
故大邦者,不過欲兼畜人;小邦者,不過欲入事人。
夫皆得其欲,則大者宜為下。

\end{withgezhu}

%
% 第二十五章
%

\chapter{~}

%节
\section{原道德經第六十二章(未校註)}

%进入割注环境
\begin{withgezhu}

%切换字体
\zhsong

%文字

道者,萬物之註也,\textcolor{tongjia-color}{善人之寶}也,不善人之所保也。
美言可以市,尊行可以加人。人之不善也,何棄之有?
故立天子,置三卿,雖有共之璧以先駟馬,不若坐而進此。
古之所以貴此道者何也?不謂求以得,有罪以免與!故為天下貴。

\end{withgezhu}

%
% 第二十六章
%

\chapter{~}

%节
\section{原道德經第六十三章(未校註)}

%进入割注环境
\begin{withgezhu}

%切换字体
\zhsong

%文字

為無為,事無事,味無味,大小,多少,報怨以德。圖難乎其易也,為大乎其細也。
天下之難作於易,天下之大作於細。是以聖人終不為大,故能成其大。
夫輕諾必寡信,多易必多難。是以聖人猶難之,故終於無難。

\end{withgezhu}

%
% 第二十七章
%

\chapter{~}

%节
\section{原道德經第六十四章(未校註)}

%进入割注环境
\begin{withgezhu}

%切换字体
\zhsong

%文字

其安也,易持也。其未兆也,易謀也。其脆也,易破也。其微也,易散也。
為之於其未亂也,治之於其未亂也。合抱之木,生於毫末。
九成之臺,作於累土。\textcolor{tongjia-color}{百仞之高,始於足下}。
為之者敗之,執之者失之。是以聖人無為也,故無敗也;無執也,故無失也。
民之從事也,恒於其成事而敗之。故慎終若始,則無敗事矣。
是以聖人欲不欲,而不貴難得之貨;學不學,復眾人之所過;能輔萬物之自然,而弗敢為。

\end{withgezhu}

%
% 第二十八章
%

\chapter{~}

%节
\section{原道德經第六十五章(未校註)}

%进入割注环境
\begin{withgezhu}

%切换字体
\zhsong

%文字

故曰:為道者非以明民也,將以愚之也。民之難治也,\textcolor{tongjia-color}{以其智也}。
\textcolor{tongjia-color}{故以智治邦,邦之賊也}。以不智治邦,邦之德也。
恒知此兩者,亦稽式也。恒知稽式,此謂玄德。玄德深矣,遠矣、與物反也,乃至大順。

\end{withgezhu}

%
% 第二十九章
%

\chapter{~}

%节
\section{原道德經第六十六章(未校註)}

%进入割注环境
\begin{withgezhu}

%切换字体
\zhsong

%文字

江海之所以能為百谷王者,以其善下之,是以能為百谷王。
是以聖人之欲上民也,必以其言下之;欲先民也,必以其身後之。
故居前而民弗害也,居上而民弗重也。\textcolor{tongjia-color}{天下樂推而弗厭也}。
非以其無爭與,故天下莫能與爭。

\end{withgezhu}

%
% 第三十章
%

\chapter{~}

%节
\section{原道德經第八十章(未校註)}

%进入割注环境
\begin{withgezhu}

%切换字体
\zhsong

%文字

小邦寡民,\textcolor{tongjia-color}{使十百人之器毋用,使民重死而遠徙},
有舟車無所乘之,有甲兵無所陳之,使民復結繩而用之。
甘其食,美其服,樂其俗,安其居,鄰邦相望,雞犬之聲相聞,民至老死不相往來。

\end{withgezhu}

%
% 第三十一章
%

\chapter{~}

%节
\section{原道德經第八十一章(未校註)}

%进入割注环境
\begin{withgezhu}

%切换字体
\zhsong

%文字

信言不美,美言不信。\textcolor{tongjia-color}{知者不博,博者不知}。
善者不多,多者不善。聖人無積,既以為人,己愈有;既以予人矣,己愈多。
故天之道,利而不害;人之道,為而弗爭。

\end{withgezhu}

%
% 第三十二章
%

\chapter{~}

%节
\section{原道德經第六十七章(未校註)}

%进入割注环境
\begin{withgezhu}

%切换字体
\zhsong

%文字

天下皆謂我大,大而不肖。夫唯不肖,故能大;若肖,久矣其細也夫。
我恒有三寶,\textcolor{tongjia-color}{持而寶之},一曰慈,二曰儉,三曰不敢為天下先。
夫慈,故能勇;儉,故能廣;不敢為天下先,故能為成器長。
今舍其慈,且勇;舍其後,且先,則必死矣。夫慈,以戰則勝,以守則固。
天將建之,汝以慈垣之。

\end{withgezhu}

%
% 第三十三章
%

\chapter{~}

%节
\section{原道德經第六十八章(未校註)}

%进入割注环境
\begin{withgezhu}

%切换字体
\zhsong

%文字

故善為士者不武,善戰者不怒,善勝敵者弗與,善用人者為之下。
是謂不爭之德,是謂用人,是謂配天;古之極也。

\end{withgezhu}

%
% 第三十四章
%

\chapter{~}

%节
\section{原道德經第六十九章(未校註)}

%进入割注环境
\begin{withgezhu}

%切换字体
\zhsong

%文字

用兵有言曰:“吾不敢為主而為客,吾不進寸而退尺。”
是謂行無行,攘無臂,執無兵,乃無敵矣。禍莫大於無敵,無敵近亡吾寶矣。
故抗兵相若,則哀者勝矣。

\end{withgezhu}

%
% 第三十五章
%

\chapter{~}

%节
\section{原道德經第七十章(未校註)}

%进入割注环境
\begin{withgezhu}

%切换字体
\zhsong

%文字

吾言甚易知也,甚易行也;而人莫之能知也,而莫之能行也。
言有君,事有宗;夫唯無知也,是以不我知?知我者希,則我貴矣。
是以聖人被褐而懷玉。

\end{withgezhu}

%
% 第三十六章
%

\chapter{~}

%节
\section{原道德經第七十一章(未校註)}

%进入割注环境
\begin{withgezhu}

%切换字体
\zhsong

%文字

知不知,尚矣;不知知,病矣。是以聖人之不病,以其病病也,是以不病。

\end{withgezhu}

%
% 第三十七章
%

\chapter{~}

%节
\section{原道德經第七十二章(未校註)}

%进入割注环境
\begin{withgezhu}

%切换字体
\zhsong

%文字

民之不畏威,則大威將至矣。毋狹其所居,毋厭其所生。夫唯弗厭,是以不厭。
是以聖人自知而不自見也,自愛而不自貴也。故去彼取此。

\end{withgezhu}

%
% 第三十八章
%

\chapter{~}

%节
\section{原道德經第七十三章(未校註)}

%进入割注环境
\begin{withgezhu}

%切换字体
\zhsong

%文字

勇於敢者則殺,勇於不敢者則活。此兩者或利或害。
天之所惡,孰知其故?天之道,不戰而善勝,不言而善應,不召而自來,坦而善謀。
天網恢恢,疏而不失。

\end{withgezhu}

%
% 第三十九章
%

\chapter{~}

%节
\section{原道德經第七十四章(未校註)}

%进入割注环境
\begin{withgezhu}

%切换字体
\zhsong

%文字

\textcolor{tongjia-color}{若民恒且不畏死,奈何以殺懼之也}?
\textcolor{tongjia-color}{若民恒且畏死,而為奇者吾得而殺之,夫孰敢矣}。
\textcolor{tongjia-color}{若民恒且必畏死,則恒有司殺者}。
夫代司殺者,是代大匠斫也。夫代大匠斫者,則希不傷其手矣。

\end{withgezhu}

%
% 第四十章
%

\chapter{~}

%节
\section{原道德經第七十五章(未校註)}

%进入割注环境
\begin{withgezhu}

%切换字体
\zhsong

%文字

人之饑也,以其取食稅之多也,是以饑。百姓之不治也,以其上有以為也,是以不治。
民之輕死也,以其求生之厚也,是以輕死。夫唯無以生為者,是賢貴生。

\end{withgezhu}

%
% 第四十一章
%

\chapter{~}

%节
\section{原道德經第七十六章(未校註)}

%进入割注环境
\begin{withgezhu}

%切换字体
\zhsong

%文字

人之生也柔弱,\textcolor{tongjia-color}{其死也筋韌堅強}。
萬物草木之生也柔脆,其死也枯槁。
故曰:堅強者死之徒也;柔弱者生之徒也。兵強則不勝,木強則烘。
故強大居下,柔弱居上。

\end{withgezhu}

%
% 第四十二章
%

\chapter{~}

%节
\section{原道德經第七十七章(未校註)}

%进入割注环境
\begin{withgezhu}

%切换字体
\zhsong

%文字

天之道,猶張弓者也。高者抑之,下者舉之;有余者損之,不足者補之。
故天之道,損有余而益不足;人之道則不然,損不足而奉有余。
孰能有余而有以取奉於天者乎?唯有道者。
是以聖人為而弗有,\textcolor{tongjia-color}{成功而弗居也},若此其不欲見賢也。

\end{withgezhu}

%
% 第四十三章
%

\chapter{~}

%节
\section{原道德經第七十八章(未校註)}

%进入割注环境
\begin{withgezhu}

%切换字体
\zhsong

%文字

天下莫柔弱於水,而攻堅強者莫之能勝,以其無以易之也。
\textcolor{tongjia-color}{柔之勝剛也,弱之勝強也},天下莫弗知也,而莫之能行也。
是故聖人之言云,曰:受邦之詬,\textcolor{tongjia-color}{是謂社稷之主};
受邦之不祥,\textcolor{tongjia-color}{是謂天下之王}。正言若反。

\end{withgezhu}

%
% 第四十四章
%

\chapter{~}

%节
\section{原道德經第七十九章(未校註)}

%进入割注环境
\begin{withgezhu}

%切换字体
\zhsong

%文字

和大怨,必有余怨,焉可以為善?\textcolor{tongjia-color}{是以聖人持右契},而不以責於人。
故有德司契,無德司徹。夫天道無親,恒與善人。
\gezhu{帛書乙本最後註:德三千零四十一}

\end{withgezhu}



%道篇
%
% 道篇
%

%篇
\part*{道篇}


\end{document}

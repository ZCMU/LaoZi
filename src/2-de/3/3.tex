%
% 第三章
%

\chapter{~}

%节
\section{原道德經第四十一章}

%进入割注环境
\begin{withgezhu}

%切换字体
\zhsong

%文字

上\textcolor{wangbi-color}{士聞}道,堇(\textcolor{tongjia-color}{勤})能用之。
中士聞道,若存若亡。下士聞道,大笑之。弗笑,\textcolor{wangbi-color}{不足}以為道。
是以建言有之曰:明道如費(\textcolor{tongjia-color}{昧}),進道如退,夷道如類\gezhu{平坦的道好像不平坦}。
上德如浴(\textcolor{tongjia-color}{谷}),大白如辱\gezhu{汙垢之意}。
廣德如不足,建德如(\textcolor{tongjia-color}{偷})。質\textcolor{wangbi-color}{真如渝}。
大方無禺(\textcolor{tongjia-color}{隅}),大器免成。
大音希聲,天象無刑(\textcolor{tongjia-color}{形}),道褒無名。
\textcolor{wangbi-color}{夫唯}道,善\textcolor{wangbi-color}{始且善成}。

\end{withgezhu}

%节
\section{註記}

\begin{withgezhu}

\zhsong

甲本此章經文大量殘毀,僅留乙本最後一句“道”和“善”二字。
今本第四十章與第四十一章順序顛倒,經研究,帛書《老子》順序正確。

\end{withgezhu}

%节
\section{王弼本}

\begin{withgezhu}

\zhsong

上士聞道,勤而行之。中士聞道,若存若亡。下士聞道,大笑。不笑,不足以為道。
故建言有之:明道如昧,進道若退,夷道若類。
上德若谷,大白若辱。廣德若不足,建德若偷。質真若渝。
大方無隅,大器免成\gezhu{後人傳訛為大器晚成},大音希聲,大象無形,道隱無名。
夫唯道,善貸且成。

\end{withgezhu}

%
% 第一章
%

\chapter{~}

%节
\section{原道德經第三十八章}

%进入割注环境
\begin{withgezhu}

%切换字体
\zhsong

%文字

\colorbox{adding-color}{上德不德,是以有德;下德不失德,是以無}德。
上德無\colorbox{adding-color}{為而}無以為也。
上仁為之\colorbox{adding-color}{而無}以為也。
上義\gezhu{乙本義字作德字}為之而有以為也。
上禮\colorbox{adding-color}{為之而莫之應也,則}
攘臂而乃(\textcolor{tongjia-color}{扔})之
\gezhu{德仁義禮四层递减}。
故失道而後德,失德而後仁,失仁而後義,
\colorbox{adding-color}{失義而後禮。}
\colorbox{adding-color}{夫禮者,}
\colorbox{adding-color}{忠信之泊(\textcolor{tongjia-color}{薄})也},
而亂之首也。
\colorbox{adding-color}{前識者},道之華也,而愚之首也。
是以大丈夫居\colorbox{missing-color}{其厚而不}居其泊(\textcolor{tongjia-color}{薄});
居其實而不居其華。故去皮(\textcolor{tongjia-color}{彼})\gezhu{乙本多一而字}取此。

\end{withgezhu}

%节
\section{王弼本}

\begin{withgezhu}

\zhsong

上德不德、是以有德;下德不失德,是以無德。
上德無為而無以為。下德為之而有以為\gezhu{此句為後人妄加}。
上仁為之而無以為。上義為之而有以為。
上禮為之而莫之應,則攘臂而扔之。
故失道而後德,失德而後仁,失仁而後義,失義而後禮。
夫禮者,忠信之薄而亂之首。前識者,道之華也,而愚之始。
是以大丈夫處其厚不居其薄;處其實不居其華。故去彼取此。

\end{withgezhu}

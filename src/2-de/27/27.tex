%
% 第二十七章
%

\chapter{~}

%节
\section{原道德經第六十四章(未校註)}

%进入割注环境
\begin{withgezhu}

%切换字体
\zhsong

%文字

其安也,易持也。其未兆也,易謀也。其脆也,易破也。其微也,易散也。
為之於其未亂也,治之於其未亂也。合抱之木,生於毫末。
九成之臺,作於累土。\textcolor{tongjia-color}{百仞之高,始於足下}。
為之者敗之,執之者失之。是以聖人無為也,故無敗也;無執也,故無失也。
民之從事也,恒於其成事而敗之。故慎終若始,則無敗事矣。
是以聖人欲不欲,而不貴難得之貨;學不學,復眾人之所過;能輔萬物之自然,而弗敢為。

\end{withgezhu}

%
% 第二章
%

\chapter{~}

%节
\section{原道德經第三十九章}

%进入割注环境
\begin{withgezhu}

%切换字体
\zhsong

%文字

昔之\gezhu{乙本缺之字}得一者,
天得一以清,地得\colorbox{adding-color}{一}以寧,
神得一以霝(\textcolor{tongjia-color}{靈}),
浴(\textcolor{tongjia-color}{谷})得一以\gezhu{乙本缺以字}盈,
侯\colorbox{adding-color}{王得一}而\gezhu{乙本缺而字}以為\colorbox{adding-color}{天下}正。
其致(\textcolor{tongjia-color}{誡})之\gezhu{乙本作至字,缺之字}也,
胃(\textcolor{tongjia-color}{謂})天毋已清將恐\colorbox{adding-color}{裂}\gezhu{乙本作蓮字,通裂},
胃(\textcolor{tongjia-color}{謂})\gezhu{乙本缺胃字}地毋\colorbox{adding-color}{已寧}將恐\colorbox{adding-color}{發},
胃(\textcolor{tongjia-color}{謂})\gezhu{乙本缺胃字}神毋\colorbox{missing-color}{已霝(\textcolor{tongjia-color}{靈})}\textcolor{wangbi-color}{將}\colorbox{adding-color}{恐歇},
胃(\textcolor{tongjia-color}{謂})\gezhu{乙本缺胃字}浴(\textcolor{tongjia-color}{谷})毋已\colorbox{missing-color}{盈}將恐渴(\textcolor{tongjia-color}{竭}),
胃(\textcolor{tongjia-color}{謂})\gezhu{乙本缺胃字}侯王毋已貴\colorbox{adding-color}{以高將恐厥(\textcolor{tongjia-color}{蹶})}。
故必貴而\gezhu{乙本缺而字}以賤為本,必高矣而以下為基。
夫是以侯王自胃(\textcolor{tongjia-color}{謂})孤寡不榖\gezhu{榖原文為左半個字,無法拼寫}(\textcolor{tongjia-color}{榖})。
此其\colorbox{adding-color}{賤之本與,非也}?
故致\gezhu{乙本作至字}數與(\textcolor{tongjia-color}{譽})\gezhu{乙本作輿字}無與(\textcolor{tongjia-color}{譽})\gezhu{乙本作輿字}。
是故不欲\colorbox{adding-color}{祿祿}若玉,硌\colorbox{adding-color}{硌若石}。

\end{withgezhu}

%节
\section{王弼本}

\begin{withgezhu}

\zhsong

昔之得一者,天得一以清,地得一以寧,神得一以靈,谷得一以盈,
萬物得一以生\gezhu{此句為後人妄加},侯王得一以為天下貞。
其致之,天無以清將恐裂,地無以寧將恐發,神無以靈將恐歇,谷無以盈將恐竭,
萬物無以生將恐滅\gezhu{此句為後人妄加},侯王無以貴高將恐蹶。
故貴以賤為本,高以下為基。是以侯王自謂孤寡不榖。此非以賤為本邪,非乎?
故致數輿無輿。不欲琭琭如玉,珞珞如石。

\end{withgezhu}

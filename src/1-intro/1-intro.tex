%
% 引言
%

%篇
\part*{引言}

%章
\chapter*{導語}
\addcontentsline{toc}{chapter}{導語}

《道德經》一書,世傳版本已證實經後人篡改而嚴重失真。
一九七三年,湖南長沙馬王堆漢墓出土帛書《老子》甲、乙本(合稱帛本《老子》),
距今兩千多年的歷史;
一九九三年湖北荊門郭店楚墓中又出土了一個版本更早的竹簡本《老子》,
通過比對,馬王堆漢墓出土的帛本《老子》,也就是後人常說的《道德經》或者《老子五千言》,
是至今世人篡改最少的版本。

\noindent
帛書《老子》順序和傳世本《道德經》不同,《德經》在前,《道經》在後。
本書正文按此順序編排,並根據三個版本相互勘校和比較。
如帛書老子因埋藏而被損壞之字,甲本殘壞則據乙本補,用淺綠色的背景文字表示;
乙本殘壞則據甲本補,用淺紅色背景文字表示;
乙本和甲本不同之處用割註表示;
甲乙版本共同殘壞則據王本或其它今本補,用紅色的文字表示。
帛書老子原有之衍文脫句等錯誤,錄文不刪不補,仍照原文寫成今字,只在割註予以辯證說明。
帛書甲、乙本經文中均有假借字和古字,勘校時除將按原形寫出後,
在其字下註明當用之本字和今字,皆用括弧括起,用藍色的文字以示區分。
